\section{Definitions}
\begin{description}
  \item[Node] A server which stores and serves parts.
  \item[Provider] An individual or organization which owns, secures, and operates nodes.
  \item[Tenant] An individual or organization which owns content.
  \item[Content] A versioned set of data which is owned by a tenant.
  \item[Space] A group of providers and tenants, where providers agree to run nodes that serve content owned by a tenant according to a common set of rules.
  \item[Part] A part is a sequence of bytes stored in the space, referenced by its hash. 
  \item[Content Object Version] A collection of parts created by a tenant, referenced by its hash.
  \item[Content Object] A collection of versions.
  \item[Library] A 'folder' of content objects owned by a tenant with a permission structure what determines who within a tenancy is able to create, modify, delete content objects and content object versions.
  \item[KMS] A tenant-owned server which holds keys for encrypting/decrypting content which the tenant stores in the space.
\end{description}

The following entities are defined by fixed length identifiers as follows:
\begin{center}
  \begin{tabular}{| c | l | l |}
    \hline
    Entity & Identifier & Substrate Type \\
    \hline 
    Node & $\nodeid$ & 10-byte array \\ %TODO: Could just be an unsigned 32-bit integer and do some round robin or mod-magic for assigning partitions
    Space & $\spaceid$ & 10-byte array \\
    Content Object & $\conqid$ & 10-byte array \\
    Content Object Version & $\versid$ & 32-byte array \\
    KMS & $\kmsid$ & 10-byte array \\
    Library & $\libid$ & unsigned 16-bit integer \\
    \hline 
  \end{tabular}
\end{center}
