\documentclass{article}
\usepackage{amsfonts}
\usepackage{enumitem}

\title{Eluvio Content Fabric V2 Spec}
\date{2022}

\begin{document}

\maketitle



\newcommand{\spaceid}{\mathtt{ID_{space}}}
\newcommand{\provid}{\mathtt{ID_{prov}}}
\newcommand{\nodeid}{\mathtt{ID_{node}}}
\newcommand{\tenantid}{\mathtt{ID_{tenant}}}
\newcommand{\conqid}{\mathtt{ID_{conq}}}
\newcommand{\versid}{\mathtt{ID_{version}}}
\newcommand{\libid}{\mathtt{ID_{lib}}}
\newcommand{\kmsid}{\mathtt{ID_{kms}}}

\newcommand{\keyid}[1]{k_\mathtt{#1}}
\newcommand{\amount}[1]{A_\mathtt{#1}}
\newcommand{\bond}[1]{\mathtt{Bond_{#1}}}
\newcommand{\perm}[1]{\mathtt{Perm_{#1}}}
\newcommand{\loc}[1]{\mathtt{Loc_{#1}}}

\newcommand{\rightstype}[0]{\mathcal{R}}
\newcommand{\rights}[1]{\mathtt{Rights_{#1}}}
\newcommand{\editrights}[1]{\mathtt{CanEdit}(#1)}
\newcommand{\adminrights}[1]{\mathtt{IsAdmin}(#1)}


\setcounter{section}{-1}
\section{Definitions}
\begin{description}
  \item[Node] A server which stores and serves parts.
  \item[Provider] An individual or organization which owns, secures, and operates nodes.
  \item[Tenant] An individual or organization which owns content.
  \item[Content] A piece of data which is owned by a tenant.
  \item[Space] A group of providers, provider nodes, and tenants, where providers agree to run nodes which store content owned by a tenant according to a common set of rules.
  \item[Part] A part is a sequence of bytes stored in the space, referenced by its hash. 
  \item[Content Object Version] A collection of parts created by a tenant, referenced by its hash.
  \item[Content Object] A collection of versions.
  \item[Library] A 'folder' of content objects owned by a tenant with a permission structure that determines who within a tenancy is able to create/modify/delete content objects and content object versions.
  \item[KMS] A tenant-owned server which holds keys for encrypting/decrypting content which the tenant stores in the space.
  \item[Blockchain] A distributed ledger responsible for orchestrating cooperation between providers, the exchange of value between providers and tenants, and governance that determines the rules of a space.
\end{description}

The following entities are defined by fixed length identifiers as follows:
\begin{center}
  \begin{tabular}{| c | l | l |}
    \hline
    Entity & Identifier & Type \\
    \hline 
    Node & $\nodeid$ & Fixed length byte string \\
    Space & $\spaceid$ & Fixed length byte string \\
    Content Object & $\conqid$ & Fixed length byte string \\
    Content Object Version & $\versid$ & Fixed length byte string \\
    KMS & $\kmsid$ & Fixed length byte string \\
    Library & $\libid$ & Fixed length byte string \\
    \hline 
  \end{tabular}
\end{center}

\pagebreak
\section{Space}

The space functions as the top level governance structure of the fabric that orchestrates how providers cooperate to serve tenant data. 

Spaces are identified by a fixed length $\spaceid$.

\subsection{Space rules}
\begin{description}
  \item[Provider Bond] An amount, $\bond{prov}$, of currency each provider must lock up in order to participate within the space. Funds can be slashed from here if a provider misbehaves.
\end{description}

\section{Provider}

A provider is a logical group of nodes within a space, and a permission structure for keys.
When a provider is created, it is bootstrapped with an admin key $\keyid{admin}$ that has total control of the provider.

\subsection{Permissions}
Provider keys have the following permission levels, from most to least privileged

\begin{enumerate}
  \item $\perm{root}$ can do everything
  \item $\perm{admin}$ can add/remove nodes, bill tenants
  \item $\perm{node}$ can co-author versions with tenants
\end{enumerate}

\subsection{Blockchain actions}
\begin{description}
  \item[Create Provider($\keyid{origin} = \keyid{admin}, \spaceid, \provid$)] Creates the provider
    \begin{itemize}
      \item TODO: check governance to see whether origin can create a provider
      \item Creates $\provid$ and sets its space to $\spaceid$
      \item Sets $\keyid{admin}$ as the creator of $\provid$
      \item Sets $\keyid{admin}$ as a key for $\provid$ with level $\perm{root}$
      \item Bonds $\bond{prov}$ from $\keyid{admin}$ to the space under $\provid$
    \end{itemize}
  \item[Add Node($\keyid{origin}, \provid, \nodeid, \keyid{node}, \loc{node}$)] adds a node
    \begin{itemize}
      \item Checks that $\keyid{origin}$ has permission $\perm{admin}$ or above for $\provid$.
      \item Creates a node $\nodeid$ with locator $\loc{node}$
      \item Registers $\keyid{node}$ to $\provid$ with permission level $\perm{node}$ \footnote{Should this error if the key already exists within the permissions scheme?}
    \end{itemize}
\end{description}

\end{document}
