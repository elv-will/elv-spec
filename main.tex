\documentclass{article}
\usepackage{amsfonts}
\usepackage{enumitem}

\title{Eluvio Content Fabric V2 Spec}
\author{Eluvio}
\date{2022}

\begin{document}

\maketitle



\newcommand{\spaceid}{\mathtt{ID_{space}}}
\newcommand{\provid}{\mathtt{ID_{prov}}}
\newcommand{\nodeid}{\mathtt{ID_{node}}}
\newcommand{\tenantid}{\mathtt{ID_{tenant}}}
\newcommand{\conqid}{\mathtt{ID_{conq}}}
\newcommand{\versid}{\mathtt{ID_{version}}}
\newcommand{\libid}{\mathtt{ID_{lib}}}
\newcommand{\kmsid}{\mathtt{ID_{kms}}}

\newcommand{\keyid}[1]{k_\mathtt{#1}}
\newcommand{\amount}[1]{A_\mathtt{#1}}
\newcommand{\bond}[1]{\mathtt{Bond_{#1}}}
\newcommand{\perm}[1]{\mathtt{Perm_{#1}}}
\newcommand{\loc}[1]{\mathtt{Loc_{#1}}}

\newcommand{\rightstype}[0]{\mathcal{R}}
\newcommand{\rights}[1]{\mathtt{Rights_{#1}}}
\newcommand{\editrights}[1]{\mathtt{CanEdit}(#1)}
\newcommand{\adminrights}[1]{\mathtt{IsAdmin}(#1)}


\setcounter{section}{-1}
\section{Definitions}
\begin{description}
  \item[Node] A server which stores and serves parts.
  \item[Provider] An individual or organization which owns, secures, and operates nodes.
  \item[Tenant] An individual or organization which owns content.
  \item[Content] A piece of data which is owned by a tenant.
  \item[Space] A group of providers, provider nodes, and tenants, where providers agree to run nodes which store content owned by a tenant according to a common set of rules.
  \item[Part] A part is a sequence of bytes stored in the space, referenced by its hash. 
  \item[Content Object Version] A collection of parts created by a tenant, referenced by its hash.
  \item[Content Object] A collection of versions.
  \item[Library] A 'folder' of content objects owned by a tenant with a permission structure that determines who within a tenancy is able to create/modify/delete content objects and content object versions.
  \item[KMS] A tenant-owned server which holds keys for encrypting/decrypting content which the tenant stores in the space.
  \item[Blockchain] A distributed ledger responsible for orchestrating cooperation between providers, the exchange of value between providers and tenants, and governance that determines the rules of a space.
\end{description}

The following entities are defined by fixed length identifiers as follows:
\begin{center}
  \begin{tabular}{| c | l | l |}
    \hline
    Entity & Identifier & Type \\
    \hline 
    Node & $\nodeid$ & Fixed length byte string \\
    Space & $\spaceid$ & Fixed length byte string \\
    Content Object & $\conqid$ & Fixed length byte string \\
    Content Object Version & $\versid$ & Fixed length byte string \\
    KMS & $\kmsid$ & Fixed length byte string \\
    Library & $\libid$ & Fixed length byte string \\
    \hline 
  \end{tabular}
\end{center}

\section{Spaces}

The space functions as the top level governance structure of the fabric that orchestrates how providers cooperate to serve tenant data. 
Governance is TBD.

\subsection{Space rules}
\begin{description}
  \item[Provider Bond] An amount, $\bond{prov}$, of currency each provider must lock up in order to participate within the space. Funds can be slashed from here if a provider misbehaves.
  \item[Tenant Bond] An amount, $\bond{ten}$, of currency each tenant must lock up in order to participate within the space. Funds can be slashed from here if a tenant misbehaves.
  \item[SLAs] Specifications for availability requirements provider nodes must have.
  \item[Partition number] The partitioning constant for part storage
\end{description}

\section{Providers}

A provider is a logical group of nodes within a space, and a permission structure for keys.

\subsection{Provider Permissions}
Provider keys have the following permission levels, from most to least privileged

\begin{enumerate}
  \item $\perm{root}$ Root level
    \begin{itemize}
      \item add/remove admins (effectively allows for admin key rotation)
    \end{itemize}
  \item $\perm{admin}$ Admin level
    \begin{itemize}
      \item add/remove nodes
      \item bill tenants
    \end{itemize}
  \item $\perm{node}$ Node level
    \begin{itemize}
      \item Co-author versions with tenants
      \item Mark itself as no longer pending
      \item Participate in part networking
    \end{itemize}
\end{enumerate}

\subsection{Provider blockchain actions}
In addition to setting permissions on keys, we have the following actions
\begin{description}
  \item[CreateProvider($\keyid{origin}, \spaceid, \provid$)] Creates the provider
    \begin{itemize}
      \item Check governance to see whether origin can create a provider
      \item Creates $\provid$ and sets its space to $\spaceid$
      \item Sets $\keyid{origin}$ as the root key ($\keyid{root}$) of $\provid$
      \item Sets $\keyid{origin}$ as a key for $\provid$ with level $\perm{root}$
      \item Bonds $\bond{prov}$ from $\keyid{root}$ to the space under $\provid$
    \end{itemize}
  \item[AddNode($\keyid{origin}, \provid, \nodeid, \keyid{node}, \loc{node}$)] adds a node
    \begin{itemize}
      \item Checks that $\keyid{origin}$ has permission $\perm{admin}$ or above for $\provid$.
      \item Creates a node $\nodeid$ with locator $\loc{node}$
      \item Registers $\keyid{node}$ to $\provid$ with permission level $\perm{node}$ \footnote{Should this error if the key already exists within the permissions scheme?}
      \item Marks the node as pending while it syncs up parts with the rest of the space
    \end{itemize}
  \item[ConfirmNode($\keyid{origin}, \provid, \nodeid$)] marks a node as no longer pending
    \begin{itemize}
      \item Checks that $\keyid{origin}$ has permissions $\perm{node}$ or above for $\provid$
      \item Sets $\nodeid$ to no longer pending
    \end{itemize}
  \item[RemoveNode($\keyid{origin}, \provid, \nodeid$)] removes a node
    \begin{itemize}
      \item Checks that $\keyid{origin}$ has permissions $\perm{admin}$ or above for $\provid$
      \item Removes all $\nodeid$ information from the space and provider
    \end{itemize}
  \item[BillTenant] TODO
\end{description}

\section{Tenants}

A tenant is an owner and creator of content. They are responsible for providing a service available to providers' nodes which can manage keys and encrypt/decrypt content.


\subsection{Tenant Permissions}
Tenant keys have the following permission levels, from most to least privileged

\begin{enumerate}
  \item $\perm{root}$ can add/remove admins
  \item $\perm{admin}$ can add/remove kmses, create libraries, and add/remove users from libraries
  \item $\perm{kms}$ can co-author content object versions with provider nodes
\end{enumerate}

\subsection{Tenant Blockchain Actions}

\begin{description}
  \item[CreateTenant($\keyid{origin} = \keyid{root}, \spaceid, \tenantid)$] creates a tenancy
    \begin{itemize}
      \item Checks governance to see whether origin can create a tenant
      \item Creates $\tenantid$ and sets its space to $\spaceid$
      \item Sets $\keyid{root}$ as the creator of $\tenantid$
      \item Sets $\keyid{root}$ as a key for $\tenantid$ with level $\perm{root}$
      \item Bonds $\bond{ten}$ from $\keyid{root}$ to the space under $\tenantid$
    \end{itemize}
  \item[AddKMS($\keyid{origin}, \tenantid, \kmsid, \keyid{kms}, \loc{kms}$)] creates a kms
    \begin{itemize}
      \item Checks that $\keyid{origin}$ has permission $\perm{admin}$ or above for $\tenantid$.
      \item Creates a KMS $\kmsid$ within $\tenantid$ with locator $\loc{kms}$
      \item Registers $\keyid{kms}$ to $\tenantid$ with permission level $\perm{kms}$
    \end{itemize}
  \item[RemoveKMS($\keyid{origin}, \tenantid, \kmsid$)] removes a node
    \begin{itemize}
      \item Checks that $\keyid{origin}$ has permissions $\perm{admin}$ or above for $\tenantid$
      \item Removes all $\kmsid$ information from the space and tenancy
      \item Removes $\keyid{kms}$ from $\tenantid$
    \end{itemize}
  \item[TODO: Remove Tenant, Top up billing balance]
\end{description}

\section{Libraries}
Libraries group keys together which may create/modify content within a specific context. 
They act as a way to separate the keys which manage tenant infrastructure (like KMSs) from keys which manage content.
The goal of the library is to have different levels of keys and different rules for different groups of content.
For example, one library could hold staging content which is not publicly accessible outside of library owners.
Content could be created there and then moved to a production library which has greater visibility.

\subsection{Library Rights}
Any \texttt{user} in a library has some associated rights, $\rights{user} \in \rightstype$. 
The exact format of this structure is TDB (currently they're bitflags), but should at the very least be able to answer the following questions:

\begin{description}
  \item[$\adminrights{\rights{user}}$:] Is the user allowed to add other users as non-admins
  \item[$\editrights{\rights{user}}$:]  Is the user allowed to edit content
\end{description}

\subsection{Library Blockchain Actions}

\begin{description}
  \item[CreateLibrary($\keyid{origin}, \tenantid, \libid, \textnormal{name}$)] creates a library
    \begin{itemize}
      \item Checks the origin has permission $\perm{admin}$ within $\tenantid$
      \item Creates a library with $\libid$ and the given name within $\tenantid$
        \begin{itemize}
          \item Note that for convenience, this call could also set $\rights{origin}$ such that $\adminrights{\rights{origin}}$ is true
        \end{itemize}
    \end{itemize}
  \item[SetRights($\keyid{origin}, \tenantid, \libid, \keyid{target}, \rights{target}$)] sets rights as a library admin
    \begin{itemize}
      \item Check that $\adminrights{\rights{origin}}$ 
      \item Checks that $!\adminrights{\rights{target}}$
      \item Sets the rights of $\keyid{target}$ in $\libid$ to $\rights{target}$
    \end{itemize}
  \item[TenantSetRights($\keyid{origin}, \tenantid, \libid, \keyid{target}, \rights{target}$)] sets rights as a tenant admin
    \begin{itemize}
      \item Check that $\keyid{origin}$ has $\perm{admin}$ in $\tenantid$
      \item Sets the rights of $\keyid{target}$ in $\libid$ to $\rights{target}$
    \end{itemize}
\end{description}

This also provides a way for other parts of the blockchain if a given key has edit access to a library.

\section{Content Objects}

\section{Key Management Services (KMSs)}

\section{Part networking}
TODO: @Serban

\end{document}
